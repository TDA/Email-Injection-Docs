\section{Languages used}

We used Python 2 to build the system. The following factors influenced our choice of language: text processing capabilities, PCRE (Perl Compatible Regular Expressions) compatibility, and the numerous libraries for HTML Parsing, HTTP request generation, mail processing etc.
We made use of the following major libraries (shown in table \ref{tab:libs}) for our system.

\begin{table}[!htbp]
	\centering
	\begin{tabular}{|c|c|}
		\hline
		\multicolumn{1}{|c|}{\textbf{Library}} &
		\multicolumn{1}{c|}{\textbf{Functionality}} \\
		\hline
		Requests & HTTP Request Generation\\
		\hline
		Beautiful Soup & HTML Parsing\\
		\hline
		Mailbox & Mail Processing\\
		\hline
		Celery & Task Queues\\
		\hline
	\end{tabular}
	\caption{Libraries that we used and their functions.}
	\label{tab:libs}
\end{table}

Despite the many benefits that Python 2 provides, we had certain issues with the language --- discussed in Section \ref{issues:parallel} --- like Python's GIL (Global Interpreter Lock) which does not allow the running of multiple native threads concurrently.
The following section (Section \ref{exp:Celery}) describes in detail the task queue system (Celery) that we used to overcome this limitation of Python.
