\section{Fuzzed data}
We performed our fuzzing attempts on the gathered data with both the regular payload and malicious payload. Table~\ref{tab:fuzzed_data} shows the quantity of e-mails that we received for each payload. We explain in detail what each piece of data shown in the table represents, in the following sections.
\begin{table}[!htbp]
	\centering
	\begin{tabular}{|c|c|c|c|}
		\hline
		\multicolumn{1}{|c|}{\textbf{S.No}} &
		\multicolumn{1}{c|}{\textbf{Type of fuzzing}} &
		\multicolumn{1}{c|}{\textbf{Forms fuzzed}} &
		\multicolumn{1}{c|}{\textbf{E-Mails received}}\\
		\hline
		1 & Regular payload & \fuzzed & \recd \\
		\hline
		2 & Malicious payload & \malfuzzed & \success \\
		\hline
	\end{tabular}
	\caption[\titlecap{Fuzzed data}]{The data that we fuzzed and the e-mails that we received.}
	\label{tab:fuzzed_data}
\end{table}
\paragraph{E-Mail received from forms}
The e-mails that we received can be broadly categorized into two categories:
\begin{enumerate}
	\item E-Mails due to regular payload\\
	This represents the total number of websites that sent e-mails to us. This indicates that we were able to successfully submit the forms on these sites.
	
	\item E-Mails due to malicious payload\\
    Once we receive an e-mail from a website due to the regular payload, we go back and fuzz those forms with more malicious payloads. This field, in essence, represents the total number of unique URLs that contain E-Mail Header Injection vulnerability.
\end{enumerate}



