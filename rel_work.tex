\chapter{Related Work}
This will be a detailed section on the papers that are related to our work, *but* important thing is to show why our work is different from prior work in this area.
Also, can/will add references to the blogs and books that describe this attack :)

There are different approaches to finding vulnerabilities in web applications, two of them being Black-Box testing and White-Box testing.
Our work is based on the black-box testing approach to finding vulnerabilities on websites, and there has been plenty of research that has made use of this methodology (\cite{Beizer:1995:BTT:202699}, \cite{Huang}, \cite{zanero2005automatic}, \cite{kals2006secubat}, \cite{payet13:ears-in-the-wild} etc.). There has been significant discussion on both the benefits of such an approach (\cite{black-box}) and its shortcomings (\cite{Doupe2010}, \cite{Doupe2012}).

Our work does not intend to act as an `everything and the kitchen sink' vulnerability scanner, but as a means to identify the presence of E-Mail Header Injection Vulnerability in a given web application. In this sense, since we are injecting payloads into the web application, our work is related to other injection based attacks, such as SQL Injection (\cite{sql0}, \cite{sql1}, \cite{sql2}), Cross-Site Scripting --- XSS --- (\cite{Injection1}, \cite{KleinAmit}), HTTP Header Injection (\cite{sessionride}), and is very closely related to Simple Mail Transfer Protocol (SMTP) Injection (\cite{Terada2015}).

As specified before, although this vulnerability has been present for over a decade, there has not been much written about it in the literature. 
//describe in detail Teradas work and why it is similar to but different from ours