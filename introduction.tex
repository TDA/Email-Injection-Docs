\chapter{Introduction}
	The World Wide Web has single handedly brought about a change in the way we use computers. The ubiquitous nature of the Web has made it possible for the general public to access it anywhere, and on multiple devices like Phones, Laptops, Personal Digital Assistants, and even on TVs and cars. While this rapid pace of development has improved the speed of dissemination of information, it does come at a cost. Attackers 
	\cite{Injection}
	This section will have a brief overview about the project. What the problem is, and What we aim to achieve.

\paragraph{Structure of document} % describes the remaining sections and gives a short desc about them
This document is divided logically into the following sections:
\begin{itemize}
	\item Chapter 2 discusses the background of E-Mail Header Injection, a brief history of the vulnerability, and proceeds onto enumerate the languages and platforms affected by this vulnerability.% describe each section 
	
	\item Chapter 3 discusses the System design, and enunciates the architecture and the components of the system, along with a detailed test plan to validate the system. It also enumerates the issues faced, and the assumptions made.
	
	\item Chapter 4 briefly describes the experimental setup and sheds light on how we overcame the issues and assumptions discussed in the previous section.
	
	\item Chapter 5 presents our findings, and our analysis of the said findings.
	
	\item Chapter 6 continues the discussion of the results, the lessons learned over the course of the project, limitations, and a suitable mitigation strategy to overcome the vulnerability.
	
	\item Chapter 7 explores related work in the area, and clearly shows how and why our research is different.
	
	\item Chapter 8 wraps up the document, with ideas to expand the research in this area.
\end{itemize} 

\paragraph{} % summary paragraph
In summary, we make the following contributions:
\begin{itemize}
	
	\item{A black-box approach to detecting the presence of E-Mail header injection vulnerability in a web application.}
	
	\item{A detection and classification tool based on the above approach, that will automatically detect such E-Mail Header Injection vulnerabilities in a web application.}
	
	\item{A quantification of the presence of such vulnerabilities on the World Wide Web, based on a expansive crawl across the Web, including {'x'} URLs and {'y'} forms.}
	
\end{itemize}