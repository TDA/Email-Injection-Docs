\begin{table}[!htbp]
	\centering
	\begin{tabular}{|p{2cm}|p{12cm}|}
		\hline
		\multicolumn{1}{|c|}{\textbf{Year}} & \multicolumn{1}{c|}{\textbf{ Notes}}\\
		\hline

		{Early 2000's } & { PHP 4.0 gets released, along with support for the mail() function, which has no protection against E-Mail Header Injection.}\\
		\hline

		{Jul 04} & { Next Major version of PHP - Version 5.0 releases}\\
		\hline

		{Dec 04} & { First known article about the vulnerability surfaces on phpsecure.info}\\
		\hline

		{2005 - 2007} & {XSS and SQL steal all the limelight from our poor E-Mail Header Injection.}\\
		\hline

		{Oct 07} & {The vulnerability makes its way into a text by Stuttard and Pinto. }\\
		\hline

		{Dec 08} & {Blog post and accompanying wiki about the header injection attack in detail with examples.}\\
		\hline

		{Apr 09} & {Bug filed about email.header package to fix the issue on Python Bug Tracker}\\
		\hline

		{Jan 11} & {Bug fix for Python 3.1, Python 3.2, Python 2.7 for email.header package, backport to older versions not available.}\\
		\hline

		{Sep 11} & {The vulnerability is described with an example in the 2nd edition of the text by Stuttard and Pinto.}\\
		\hline

		{Aug 13} & {Acunetix adds E-Mail Header Injection to the list of vulnerabilties they detect, as part of their Enterprise Web Vulnerability Scanner Software.}\\
		\hline

		{May 14} & {Security Advisory for JavaMail SMTP Header Injection via method setSubject is written by Alexandre Herzog.}\\
		\hline

		{Dec 15}  & {PHP 7 releases, mail function still unpatched.}\\
		\hline
	\end{tabular}
	\caption{A brief history of E-Mail Header Injection}
	\label{tab:history}
\end{table}
