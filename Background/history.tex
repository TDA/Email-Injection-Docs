\section{History of E-Mail Injection}

E-Mail Header Injection seems to have been first documented over a decade ago, in a late 2004 Article on phpsecure.info (\cite{Tobozo}) accredited to user tobozo@phpsecure.info describing how this vulnerability existed on the reference implementation of the mail function in PHP, and how it can be exploited. More recently, a blog post by Damon Kohler (\cite{DK}) and an accompanying wiki article (\cite{Injection}) describe the attack vector, and outline a few defense measures for the same.

Since this vulnerability was initially found in the \emph{mail()} function of PHP, E-Mail Header Injection can be traced to as early as early 2000's, present in the \emph{mail()} implementation of PHP 4.0. It is to be noted that after 13 further iterations of the language (the current version is 7.1), the \emph{mail()} function is yet to be fixed.

The vulnerability was also described very briefly (less than a page) by Stuttard and Pinto in their widely acclaimed book, ``\emph{The Web Application Hacker's Handbook: Discovering and Exploiting Security Flaw}'' (\cite{stuttard2011web}). 
The closest published works on this topic are a whitepaper for MBSD by Terada [6], which describes a similar vulnerability, namely SMTP Injection, to attack underlying SMTP servers, and by Teodoro et al. [7], where the vulnerability is briefly discussed along with other vulnerabilities.
A timeline of the vulnerability is presented in Table \ref{tab:history}.

\begin{table}[!htbp]
	\centering
	\begin{tabular}{|p{2cm}|p{10cm}|}
		\hline
		Year & Notes\\
		\hline
		{Early 2000's} & {PHP 4.0 gets released, along with support for the mail() function, which has no protection against E-Mail Header Injection.}\\
		\hline
		{Jul 2004} & {Next Major version of PHP - Version 5.0 releases}\\
		\hline
		{Dec 2004} & {First known article about the vulnerability surfaces on phpsecure.info}\\
		\hline		
	\end{tabular}
	\caption{A brief history of E-Mail Header Injection}
	\label{tab:history}
\end{table}