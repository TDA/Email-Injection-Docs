\section{History of E-Mail Injection}

E-Mail Header Injection seems to have been first documented over a decade ago, in a late 2004 article on phpsecure.info \cite{Tobozo} accredited to user \lstinline|tobozo@phpsecure.info| describing how this vulnerability existed in the reference implementation of the \dq{\texttt{mail}} function in PHP, and how it can be exploited. More recently, a blog post by Damon Kohler \cite{DK} and an accompanying wiki article \cite{Injection} describe the attack vector and outline a few defense measures for the same.

As this vulnerability was initially found in the \emph{mail()} function of PHP, E-Mail Header Injection can be traced to as early as the beginning of the 2000's, present in the \emph{mail()} implementation of PHP 4.0. 

The vulnerability was also described briefly (less than a page) by Stuttard and Pinto in their widely acclaimed book, ``\emph{The Web Application Hacker's Handbook: Discovering and Exploiting Security Flaws}'' \cite{stuttard2011web}. 
A concise timeline of the vulnerability is presented in Table~\ref{tab:history}.

An example of the vulnerable code written in PHP is shown in Listing~\ref{code:phpemi}. This code takes in user input from the PHP superglobal \dq{\texttt{\$\_REQUEST[\textquotesingle email\textquotesingle]}}, and stores it in the variable \dq{\texttt{\$from}}, which is later passed to the \dq{\texttt{mail}} function to construct and send the e-mail.
\lstset{language=PHP,caption={PHP program with the vulnerability.},label={code:phpemi}}
\begin{lstlisting}
$from = $_REQUEST['email'];
$subject = "Hello Sai Pc";
$message = "We need you to reset your password";
$to = "schand31@asu.edu";

// example attack string to be injected as the value for
// $_REQUEST['email'] => 'sai@sai.com\nCC:spc@spc.com'
$retValue = mail($to, $subject, $message, "From: $from");
// E-Mail gets sent to both schand31@asu.edu AND spc@spc.com
\end{lstlisting}
	
When this code is given the malicious input \dq{\texttt{\lstinline{sai@sai.com\\nBCC:spc@spc.com}}} as the value of the \dq{\texttt{\$\_REQUEST[\textquotesingle email\textquotesingle]}}, it generates the SMTP Headers shown in Listing~\ref{code:smtpheaders}. It can be seen that the `CC' (carbon copy) header that we injected appears as part of the resulting SMTP message. This will make the e-mail get sent to the e-mail address specified as part of the `CC' as well.
\lstset{caption={Generated SMTP headers.},label={code:smtpheaders}}
\begin{lstlisting}
Received: from mail.ourdomain.com ([62.121.130.29])
	by sai.com (Postfix) with ESMTP id 5A08E52C0154
	for <sai@sai.com>; Sun, 20 Mar 2016 13:56:58 -0700 (MST)
To: sai@sai.com
Subject: Hello Sai Pc
CC: spc@spc.com
Date: Sun, 20 Mar 2016 13:56:58 -0700 (MST)

We need you to reset your password
\end{lstlisting}
\begin{table}[!htbp]
	\centering
	\begin{tabular}{|p{2cm}|p{12cm}|}
		\hline
		\multicolumn{1}{|c|}{\textbf{Year}} & \multicolumn{1}{c|}{\textbf{ Notes}}\\
		\hline

		{Early 2000's } & { PHP 4.0 is released, along with support for the mail() function, which has no protection against E-Mail Header Injection.}\\
		\hline

		{Jul 2004} & { Next Major version of PHP - Version 5.0 releases}\\
		\hline

		{Dec 2004} & { First known article about the vulnerability surfaces on phpsecure.info}\\
		\hline

		{Oct 2007} & {The vulnerability makes its way into a text by Stuttard and Pinto. }\\
		\hline

		{Dec 2008} & {Blog post and accompanying wiki about the header injection attack in detail with examples.}\\
		\hline

		{Apr 2009} & {Bug filed about email.header package to fix the issue on Python Bug Tracker}\\
		\hline

		{Jan 2011} & {Bug fix for Python 3.1, Python 3.2, Python 2.7 for email.header package, backport to older versions not available.}\\
		\hline

		{Sep 2011} & {The vulnerability is described with an example in the 2nd edition of the text by Stuttard and Pinto.}\\
		\hline

		{Aug 2013} & {Acunetix adds E-Mail Header Injection to the list of vulnerabilties they detect, as part of their Enterprise Web Vulnerability Scanner Software.}\\
		\hline

		{May 2014} & {Security Advisory for JavaMail SMTP Header Injection via method setSubject is written by Alexandre Herzog.}\\
		\hline

		{Dec 2015}  & {PHP 7 releases, mail function still unpatched.}\\
		\hline
	\end{tabular}
	\caption[\titlecap{A brief history of e-mail header injection}]{A brief history of e-mail header injection.}
	\label{tab:history}
\end{table}

