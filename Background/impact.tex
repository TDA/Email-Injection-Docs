\section{Potential Impact}

The impact of the vulnerability can be pretty far-reaching.
Table~\ref{tab:usage} shows the current Server side language usage statistics on the Web, compiled from \cite{W3techs}. 
PHP, Java, Python and Ruby (combined) account for over 85\%\footnote{Note: a website may use more than one server-side programming language} of the websites in existence. The vulnerability can be exploited to do potentially any of the following:


\begin{itemize}
	\item Phishing and Spoofing Attacks\\
    Phishing (a variation of spoofing) refers to an attack where the recipient of an E-Mail is made to believe that the E-Mail is a legitimate one. The E-Mail usually redirects them to a malicious website, which then steals their credentials. 
    
    E-Mail Header Injection gives attackers the ability to inject arbitrary headers into an E-Mail sent by a website and control the output of the E-Mail. This adds credibility to the generated E-Mail, and can result in more successful phishing attacks.
	
	\item Spam Networks\\
	Spam networks can capitalize on the ability to send a large amount of E-Mail from servers that are trusted. By adding additional `cc' or `bcc' headers to the generated E-Mail, attackers can easily achieve this effect. 
	
	Due to the E-Mails being from trusted domains, E-Mail clients might not flag them as `spam'. If they do flag them as `spam', then that can lead to the website getting blacklisted as a spam generator. Irrespective of the behavior of spam filters, this does not bode well for the website.
	
	\item Information Extraction of legitimate users\\
	E-Mails can contains sensitive data that is meant to be accessed only by the user. Due to E-Mail Header Injection, an attacker can easily add a `bcc' header, and send the E-Mail to himself, thereby extracting important information.
	User Privacy can thus be compromised, and loss of private information can by itself lead to a host of other attacks.
	
	\item Denial of service by attacking the underlying mail server\\
    Denial of service attacks (DoS as they are popularly known), can also be aided by E-Mail Header Injection. The ability to send hundreds of thousands of E-Mails by just injecting one header field can result in overloading the mail server, and cause crashes and/or instability.
\end{itemize}

It is evident that if proper validation for E-Mail is not performed by these sites, this can quickly escalate to a huge issue.

\begin{table}[!htbp]
	\centering
	\begin{tabular}{|p{4cm}|p{4cm}|}
		\hline
		\multicolumn{1}{|c|}{\textbf{Server Side Language}} & \multicolumn{1}{c|}{\textbf{\% of Usage}}\\
		\hline
		PHP & 81.9\\
		\hline    
		ASP.NET & 15.8\\
		\hline
		Java & 3.1\\
		\hline
		Ruby & 0.6\\
		\hline
		Perl & 0.5\\
		\hline
		JavaScript & 0.2\\
		\hline
		Python & 0.2\\
		\hline
		
	\end{tabular}
	\caption{Language Usage Statistics compiled from W3Techs.}
	\label{tab:usage}
\end{table}