\section{Potential Impact}

The impact of the vulnerability can be pretty far-reaching.
Table~\ref{tab:usage} shows the current server-side language usage statistics on the Web \cite{W3techs}. 

\begin{table}[!tb]
	\centering
	\begin{tabular}{|p{4cm}|p{4cm}|}
		\hline
		\multicolumn{1}{|c|}{\textbf{Server Side Language}} & \multicolumn{1}{c|}{\textbf{\% of Usage}}\\
		\hline
		PHP & 81.9\\
		\hline    
		ASP.NET & 15.8\\
		\hline
		Java & 3.1\\
		\hline
		Ruby & 0.6\\
		\hline
		Perl & 0.5\\
		\hline
		JavaScript & 0.2\\
		\hline
		Python & 0.2\\
		\hline
		
	\end{tabular}
	\caption[\titlecap{Language usage statistics}]{Language usage statistics compiled from w3techs \cite{W3techs}.}
	\label{tab:usage}
\end{table}

PHP, Java, Python, and Ruby (combined) account for over 85\%\,\footnotemark{} of the websites measured. The vulnerability can be exploited to do potentially any of the following:

\footnotetext{A website may use more than one server-side programming language}

\begin{itemize}
	\item Phishing and Spoofing Attacks\\
    Phishing \cite{wiki:Phishing} (a variation of spoofing \cite{wiki:Spoofing_attack}) refers to an attack where the recipient of an e-mail is made to believe that the e-mail is a legitimate one. The e-mail usually redirects them to a malicious website, which then steals their credentials. 
    
    E-Mail Header Injection gives attackers the ability to inject arbitrary headers into an e-mail sent by a website and control the output of the e-mail. This adds credibility to the generated e-mail, as it is sent right from the websites and people are more ready to trust e-mail that is received from the website directly and can thus result in more successful phishing attacks.
	
	\item Spam Networks\\
	Spam networks can use E-Mail Header Injection vulnerabilities on the ability to send a large amount of e-mail from servers that are trusted. By adding additional \dq{\texttt{cc}} or \dq{\texttt{bcc}} headers to the generated e-mail, attackers can easily achieve this effect. 
	
	Due to the e-mails being from trusted domains, recipient e-mail clients might not flag them as spam. If they do flag them as spam, then that can lead to the website being blacklisted as a spam generator. 
	
	\item Information Extraction of legitimate users\\
	E-Mails can contains sensitive data that is meant to be accessed only by the user. Due to E-Mail Header Injection, an attacker can easily add a \dq{\texttt{bcc}} header, and send the e-mail to himself, thereby extracting important information.
	User privacy can thus be compromised, and loss of private information can by itself lead to other attacks.
	
	\item Denial of service by attacking the underlying mail server\\
    Denial of service attacks (DoS), can also be aided by E-Mail Header Injection. The ability to send hundreds of thousands of e-mails by just injecting one header field can result in overloading the mail server, and cause crashes and/or instability. 
\end{itemize}

It is evident that E-Mail Header Injection is a critical vulnerability that web applications must address.