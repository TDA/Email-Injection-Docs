\section{Problem Background}

E-Mail Header Injection belongs to a broad class of Vulnerabilities known simply as Injection attacks. However, unlike its more popular siblings, SQL injection (\cite{sql0}, \cite{sql1}, \cite{sql2}), cross-site scripting (XSS) (\cite{Injection1} \cite{KleinAmit}) or even HTTP Header Injection (\cite{sessionride}), relatively little research is available on E-Mail Header Injection.

As with other vulnerabilities in this class, E-Mail Header Injection is caused due to improper sanitization (or lack thereof) of user input. If the mailing script fails to check for the presence of E-Mail headers in the form fields that take in user input to send E-Mails, a malicious user, using a well-crafted payload, can control the headers set for this particular E-Mail. Suffice it to say that this can be leveraged to do a host of malicious attacks, including, but not limited to, spoofing, phishing, etc.