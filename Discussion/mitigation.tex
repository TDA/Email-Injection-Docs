\section[Mitigation Strategy]{How to prevent this attack}
This section describes the most common measures that can be taken to prevent the occurrence of this vulnerability, or at least reduce the impact.
\begin{itemize}
	\item Use Mail Libraries\\
	This is the preferred way of combating this vulnerability. Using a library that is well tested can remove the burden of input sanitization from the developer. Also, since most of these libraries are open-source, bugs are identified quicker and fixes are readily available.
	A list of known secure libraries for each popular language and framework is shown in Table \ref{tab:maillib}
	
	Using libraries such as PEAR Mail, PHPMailer, Apache Commons E-Mail, Contact Form 7, etc. can significantly reduce the occurrence of this type of attack.
	
	\begin{table}[!htbp]
		\centering
		\begin{tabular}{|c|c|}
			\hline
			Language & Mail Libraries\\
			\hline
			PHP & {{PEAR Mail\footnote{PEAR Mail Website: https://pear.php.net/package/Mail}, PHPMailer\footnote{PHPMailer Website: https://github.com/PHPMailer/PHPMailer}, Swiftmailer\footnote{Swiftmailer Website: http://swiftmailer.org/}}}\\
			\hline
			Python & PythonMail\\
			\hline
			Java & Apache Commons E-Mail\footnote{https://commons.apache.org/proper/commons-email/}\\
			\hline
			Ruby & Ruby\\
			\hline
			WordPress & Contact Form 7\footnote{https://wordpress.org/plugins/contact-form-7/}\\
			\hline
		\end{tabular}
		\caption{Mail Libraries to prevent E-Mail Header Injection}
		\label{tab:maillib}
	\end{table}
	
	\item CMS
	\item Input Validation
\end{itemize}


