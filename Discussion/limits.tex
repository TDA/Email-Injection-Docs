\section[Limitations]{Limitations of the Project}
	This section complements Section \ref{sys:issues}, and discusses the limitations of our project. The following list, although not exhaustive, goes into the limitations of our project in detail: 
	\begin{itemize}
		\item CAPTCHAs - As noted in section \ref*{issues:captcha}, CAPTCHAs pose a significant problem to our automated system. Since CAPTCHAs are robust, there is no easy way to break them. There has been considerable research in this area (\cite{captchas}, \cite{captchas2} to name a few) and although not impossible to break, it remains out of the scope of this project, and thus, we chose just to ignore the websites which require CAPTCHA verification.
		\item JavaScript Apps - Due to the emergence of Node.js as a server-side language, and the growing emphasis on responsive web applications, more and more applications are being built purely with JavaScript. Even conventional applications are now making use of JavaScript to dynamically insert content and update the pages. It might not be immediately clear, but what this means for the web application is that these dynamically injected components are not a part of the source code that is sent by the web server.
		
		Thus, our system never receives dynamically injected forms from the web server and hence is unable to detect whether these vulnerabilities are present in such forms. The only workaround would be, ironically, to use JavaScript to query for the \lstinline|document.getElementsByTagName('html')[0].innerHTML| (from inside web browser automation tools like Selenium, etc.), and then use that as the source code for our URL.
		
		Since this would add unnecessary bulk and complexity to our application, we chose not to do it, and thus, we consider this to be a limitation.
		\item Blogs powered by WordPress/Drupal\\
		In addition to what was discussed in Section \ref{issues:cms}, we found that certain WordPress plugins also prevent the E-Mail Header Injection attack by sanitizing user input on Contact Forms. Some of these are discussed in the following section.
		
		\item Blacklisting by websites and ISPs\\
		During the actual crawl, we ended up getting blacklisted by a few websites, and Internet Service Providers (ISPs). We then had to create up a blacklist of our own to ensure that we did not inject these websites.
		
		\item E-Mail libraries\\
        E-Mail libraries like the PHP Extension and Application Repository's (PEAR) Mail Library provide inbuilt sanitization checks for user input. While this is technically not a limitation of our project, it still makes it such that we are not able to inject these sites successfully, and that is enough to justify its inclusion in the limitations section.
        A few other libraries for each language are discussed in the following section.
	\end{itemize}