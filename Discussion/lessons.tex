\section{Lessons Learned}
    From our results, it is evident that the vulnerability exists in the wild. Despite its relatively low occurrence rate compared to the more popular SQL Injection and XSS (Cross-Site Scripting), when we take the total number of websites on the World Wide Web and calculate the percentage of that number, we end up with a pretty significant number. We agree that an extrapolation of that kind might not be a good measure of the prevalence of the vulnerability. However, even with as few as a thousand websites affected by this vulnerability, it can still have a disastrous impact on the global Web. 
    //The car parking analogy
    
    As mentioned many times previously, this vulnerability can have some major consequences, the least of which can be spamming and phishing attacks. In today's digital world, identity theft has become all the more common. E-Mail Header Injection provides attackers with the ability to extract easily information about users, not just from a server, but from the user himself, by sending him fake messages that look extremely authentic, since these messages are sent by the website itself.
    
    From our research, we found two different forms of the E-Mail Header Injection Vulnerability: the first one is the traditional one, where we are able to inject any header into the forms, allowing us complete control over the contents of the E-Mail. We identify this with the presence of both the `bcc' header and the `x-check' header. This is the most potent form of the vulnerability and is found on quite a few websites. This is also the vulnerability that is documented and discussed on many websites.
    
    The second attack is an interesting one, as this has not yet been documented, and provides the ability to inject multiple E-Mail addresses into only the `TO' field. In this form of the vulnerability, we are able to simply add addresses to the `To' field of the form with newlines separating the E-Mail addresses. Whether this particular form of the vulnerability is found due to the websites in question, or whether this is an implementation issue with a particular language or framework, is unclear. However, from our preliminary analysis, it is evident that these websites do not share much with respect to the languages and frameworks used. 
    Even in this form of the attack, we are still able to extract information that should be private to a given user, and in most of these cases, able to inject enough data to spoof the first few lines of the E-Mail message.
    
    While not being as lethal as the primary vulnerability, this second form of the vulnerability does still provide the ability to send the E-Mails to multiple recipients, and can easily result in information leakage, and/or spam generation.
    
