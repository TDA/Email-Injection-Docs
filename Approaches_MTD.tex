\section{Related Work }
The idea and philosophy of MTD, which is to increase uncertainty and complexity for attackers, has been proposed and studied for decades~\cite{avizienis1977implementation,ammann1988data,pettis1990profile,forrest1997building}.

Okhravi \textit{et al.} surveyed techniques that applied the philosophy of MTD in different cyber research domains~\cite{okhravi2013survey}.
According to them, existing techniques can be categorized into five classes based on what component to move:
\begin{enumerate}
	\item Changing the application environment~\cite{team2003pax,barrantes2003randomized}
	\item Changing application code dynamically or diversifying software~\cite{wartell2012binary,larsen2014sok}
	\item Changing the representation of data~\cite{ammann1988data,nguyen2008security}
	\item Changing the properties of platforms~\cite{williams2009security,salamat2011runtime}
	\item Changing the network configurations~\cite{Zhuang2013investigating,ge2014toward,jafarian2014openflow}
\end{enumerate}

% \paragraph{Explain item 1, give example}
Application environment randomization involves modifying the environment presented to the application by the system at run-time. These techniques modify configuration components such as data and instruction memory locations, heap/stack configuration, and the application's instruction set. Techniques that fall within this category typically prevent injection-attacks that seek to control the application by injecting malicious code or otherwise. 

Address Space Layout Randomization (ASLR)~\cite{team2003pax} and Instruction Set Randomization (ISR) are widely adopted instances of application environment randomization in modern operating systems.
Existing ASLR mechanisms randomly arrange the address space positions of key data areas such as the base executable memory location, application stack and heap, and any libraries it requires. (what-to-move) of a process when it is launched (when-to-move), including the base of the executable and the positions of the stack, heap, and libraries.

As a result, if an attacker manages to exploit some memory corruption vulnerability in the application binary, i.e. a buffer overflow attack, it would be difficult for attackers to transfer control flow to their injected code as they will be unable to accurately predict the application's memory layout. 

\paragraph{Explain item 2, give example}
Code diversification - High level vs. Low level

Dynamic application randomization or code diversification involves X, seeks to prevent vulnerability Y, main drawback is Z
\paragraph{Explain item 3, give example}
Changing the application's data representation involves X, seeks to prevent vulnerability Y, main drawback is Z
\paragraph{Explain item 4, give example}
Randomizing application's platform property involves X, seeks to prevent vulnerability Y, main drawback is Z
\paragraph{Explain item 5, give example}
Network configuration randomization involves X, can prevent vulnerability Y, main drawback is Z

%Introduce further classification of techniques: 
%Static and Dynamic; 
%Reactive and Proactive;
\paragraph{Further classification of techniques based on Cyber Kill Chain - Lockheed cyber kill chain}
Reconnaissance
Delivery 
Exploitation

\paragraph{Static and Dynamic MTD techniques - Static is required to be done before run-time, Dynamic can be done during run-time}
In addition to the 5 classes of MTD techniques, each class can be further categorized. For instance, MTD mechanisms for programs can be categorized into two classes depending on if a program is running (\textit{dynamic}) or not (\textit{static}) at the time when moving happens.
For instance, existing ASLR approaches are static, because the positions of code and data areas are only moved at the launch of a program but not when a program is running.
On the other-hand, dynamic MTD techniques offer a wider option of choices for when-to-move,
in exchange for being more difficult to implement due to other considerations: i.e. overhead cost and downtime during movement.