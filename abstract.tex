\begin{abstract}
	E-mail header injection vulnerability is a class of vulnerability that has been around for a long time but has not made its way to popular literature. It can be considered as the email equivalent of HTTP Header Injection Vulnerability. Email injection is possible when the mailing script fails to check for the presence of email headers in the form fields that take in email addresses. The vulnerability exists in the reference implementation of the “mail” function in popular languages like PHP and python. With the proper injection string, this vulnerability can be exploited to inject additional headers and/or modify existing headers in an E-mail message.
	\paragraph{}
	To understand and quantify the prevalence of E-Mail Header Injection vulnerabilities, we used a black-box testing approach, where we crawled {'x'} % need to fill
	URLs in order to find the URLs which contained form fields. Our system used this data feed to classify the forms which had e-mail fields which could be fuzzed with malicious payloads. Amongst the {'s'} % need to fill
	forms fuzzed, our system  was able to find {'y'} % need to fill
	 vulnerable URLs among {'z'} domains, which proves that the threat {is/isn't} % need to edit
	 widespread {and deserves future research attention.} % need to edit

\end{abstract}
